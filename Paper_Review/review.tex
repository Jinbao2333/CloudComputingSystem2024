\documentclass[
]{article}
\usepackage{amsmath,amssymb}
\usepackage{iftex}
\ifPDFTeX
  \usepackage[T1]{fontenc}
  \usepackage[utf8]{inputenc}
  \usepackage{textcomp} % provide euro and other symbols
\else % if luatex or xetex
  \usepackage{unicode-math} % this also loads fontspec
  \defaultfontfeatures{Scale=MatchLowercase}
  \defaultfontfeatures[\rmfamily]{Ligatures=TeX,Scale=1}
\fi
\usepackage{lmodern}
\ifPDFTeX\else
  % xetex/luatex font selection
\fi
% Use upquote if available, for straight quotes in verbatim environments
\IfFileExists{upquote.sty}{\usepackage{upquote}}{}
\IfFileExists{microtype.sty}{% use microtype if available
  \usepackage[]{microtype}
  \UseMicrotypeSet[protrusion]{basicmath} % disable protrusion for tt fonts
}{}
\makeatletter
\@ifundefined{KOMAClassName}{% if non-KOMA class
  \IfFileExists{parskip.sty}{%
    \usepackage{parskip}
  }{% else
    \setlength{\parindent}{0pt}
    \setlength{\parskip}{6pt plus 2pt minus 1pt}}
}{% if KOMA class
  \KOMAoptions{parskip=half}}
\makeatother
\usepackage{xcolor}
\ifLuaTeX
  \usepackage{luacolor}
  \usepackage[soul]{lua-ul}
\else
  \usepackage{soul}
\fi
\setlength{\emergencystretch}{3em} % prevent overfull lines
\providecommand{\tightlist}{%
  \setlength{\itemsep}{0pt}\setlength{\parskip}{0pt}}
\setcounter{secnumdepth}{-\maxdimen} % remove section numbering
\ifLuaTeX
  \usepackage{selnolig}  % disable illegal ligatures
\fi
\IfFileExists{bookmark.sty}{\usepackage{bookmark}}{\usepackage{hyperref}}
\IfFileExists{xurl.sty}{\usepackage{xurl}}{} % add URL line breaks if available
\urlstyle{same}
\hypersetup{
  hidelinks,
  pdfcreator={LaTeX via pandoc}}

\usepackage{geometry} % Add the geometry package
\geometry{margin=1in} % Set all margins to 1 inch (2.54 cm)

\author{}
\date{}

\begin{document}

\section{Summary of Paper No.31: RESCU-SQL: Oblivious Querying for the
Zero Trust
Cloud}\label{summary-of-paper-no31-rescu-sql-oblivious-querying-for-the-zero-trust-cloud}

Student Name: \textbf{Jiaqi Jiang} Student ID: \textbf{10225501447}

\begin{center}\rule{0.5\linewidth}{0.5pt}\end{center}

\subsubsection{\texorpdfstring{\textbf{1 Research
Context}}{1 Research Context}}\label{1--research-context}

Cloud services are increasingly used for DBMS applications due to their
availability and cost-effectiveness. However, high-stakes applications,
such as defense and healthcare, face challenges due to confidentiality
and regulatory requirements, making it difficult to trust third-party
\ul{c}loud \ul{s}ervice \ul{p}roviders (\textbf{CSP}s) with sensitive
data.

\subsubsection{\texorpdfstring{\textbf{2 Main Problem
Addressed}}{2 Main Problem Addressed}}\label{2--main-problem-addressed}

The paper addresses the challenge of securely querying databases in a
zero-trust cloud environment, where users cannot trust CSPs with their
private data, as it impacts the adoption of cloud services in sensitive
domains where data breaches can have severe consequences.

\subsubsection{\texorpdfstring{\textbf{3 Core Idea and
Methods}}{3 Core Idea and Methods}}\label{3--core-idea-and-methods}

\paragraph{3.1 Core Idea}\label{31--core-idea}

The core idea is the development of RESCU-SQL, a platform that allows
resilient and secure SQL querying on untrusted cloud servers. The system
leverages secure \ul{m}ulti\ul{p}arty \ul{c}omputation (\textbf{MPC}) to
ensure that queries can be executed without revealing private data to
the cloud servers.

\paragraph{3.2 Technics and Methods}\label{32--technics-and-methods}

Technical methods include the use of an authenticated garbling protocol
extended to the outsourced setting. A trusted coordinator generates and
distributes authenticated secret shares to reduce communication and
memory costs, making the protocol efficient.

\subsubsection{\texorpdfstring{\textbf{4 Reviews: Innovativeness and
Originality}}{4 Reviews: Innovativeness and Originality}}\label{4--reviews-innovativeness-and-originality}

Combining MPC with a trusted coordinator to achieve secure SQL querying
in a zero-trust cloud environment. The approach is original in its
extension of the authenticated garbling protocol to an outsourced
setting and its ability to tolerate up to \( n-1 \) corrupted servers.

\subsubsection{\texorpdfstring{\textbf{5 Strengths \&
Weaknesses}}{5 Strengths \& Weaknesses}}\label{5--strengths--weaknesses}

\paragraph{5.1 Strengths}\label{51--strengths}

\begin{itemize}
\item
  \textbf{Robust Security Model}

  \begin{itemize}
  \item
    The paper provides a comprehensive security analysis demonstrating
    the robustness of RESCU-SQL against various attack vectors,
    including passive and active adversaries.
  \item
    \(\boxed{\text{Commentary}}\): The thoroughness of the security
    analysis is crucial for high-stakes applications where even minor
    vulnerabilities can have significant consequences. This depth of
    analysis builds confidence in the system\textquotesingle s security
    guarantees.
  \end{itemize}
\item
  \textbf{Minimized Trusted Computing Base (TCB)}

  \begin{itemize}
  \item
    By limiting the trusted computing base to a small, trusted
    coordinator, the system reduces the trust requirements on the cloud
    servers.
  \item
    \(\boxed{\text{Commentary}}\): A minimized TCB enhances the overall
    security posture of the system, as fewer components need to be
    trusted and protected. This design choice also simplifies the
    security audit process.
  \end{itemize}
\item
  \textbf{Scalability}

  \begin{itemize}
  \item
    The system is designed to scale with the number of cloud servers,
    making it suitable for large-scale deployments.
  \item
    \(\boxed{\text{Commentary}}\): Scalability is crucial for real-world
    applications, especially in environments with large datasets and
    complex queries.
  \end{itemize}
\end{itemize}

\paragraph{5.2 Weaknesses}\label{52--weaknesses}

\begin{itemize}
\item
  \textbf{Performance Overhead}

  \begin{itemize}
  \item
    The system incurs a substantial performance overhead compared to
    plaintext querying, with slowdowns of several orders of magnitude.
  \item
    \(\boxed{\text{Commentary}}\): While security is enhanced, the
    performance trade-off might limit the adoption in scenarios where
    query speed is critical. For example, real-time analytics or
    interactive querying might be infeasible with the current
    performance overhead.
  \end{itemize}
\item
  \textbf{Complexity of Implementation}

  \begin{itemize}
  \item
    The system\textquotesingle s reliance on a trusted coordinator and
    MPC protocols introduces complexity in implementation and
    deployment.
  \item
    \(\boxed{\text{Commentary}}\): The implementation complexity of
    RESCU-SQL is significant due to the intricate cryptographic
    protocols, the need for secure infrastructure, and the challenges of
    integrating with existing systems while maintaining performance. For
    example, it is very hard to modify existing DBMS to support secure
    querying without compromising performance.
  \end{itemize}
\end{itemize}

\end{document}